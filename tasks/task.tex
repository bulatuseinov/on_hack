\documentclass[11pt]{article}
\usepackage[T2A]{fontenc} % кодировка
\usepackage[utf8]{inputenc} % кодировка исходного текста
\usepackage[english,russian]{babel} % локализация и переносы
\newcommand{\numpy}{{\tt numpy}}    % tt font for numpy
\usepackage{hyperref} 
\usepackage{xcolor}
\usepackage{pgf}
\usepackage{tikz}
\usepackage[utf8]{inputenc}
\usetikzlibrary{arrows,automata}
\usetikzlibrary{positioning}


\tikzset{
    state/.style={
           rectangle,
           rounded corners,
           draw=black, very thick,
           minimum height=2em,
           inner sep=2pt,
           text centered,
           },
}


\begin{document}

\title{Кабанчик Колокольчик}
\maketitle

\section*{Планы на подготовку}

\href{https://vk.com/@vkappsdev-quick-start}{\textcolor{blue}{Быстрыф старт с VK Mini App}}. Также всем необходимо стать админами в \href{vk.com/app7119135}{\textcolor{blue}{VK Mini App}} и форкнуть \href{https://github.com/maximzubkov/on_hack}{\textcolor{blue}{GitHub}}
\section{Распознавание голоса, {Булат, Максим, Саша}}

\

Задача распознавания голоса является одной из наиболее сложных, в проекте, поэтому желательно максимально подготовиться и посмотреть как можно больше методов. В идеале хочется, чтобы на все подзадачи уже имелись готовые решения и датасеты.

Пайплайн будет следующий:

\begin{center}
\begin{tikzpicture}[->,>=stealth']

 % First node
 % Use previously defined 'state' as layout (see above)
 % use tabular for content to get columns/rows
 % parbox to limit width of the listing
 \node[state] (soundGet) 
 {\begin{tabular}{l}
  \textbf{Получение голосовухи \href{https://www.npmjs.com/package/@cleandersonlobo/react-mic}{\textcolor{blue}{React Mic}}}

 \end{tabular}};
 
 % js lib
 \node[state, below right=1.5cm and 1cm of soundGet] (recogn) 
 {\begin{tabular}{l}
  \textbf{\href{https://www.npmjs.com/package/react-speech-recognition}{\textcolor{blue}{React speech recognition}}}\\
 \end{tabular}};
 % {Отчистка звука}
 \node[state,       % layout (defined above)
 node distance=1.2cm,     % distance to QUERY
 below of=soundGet        % Position is to the right of QUERY
] (soundClear)    % move 3cm in y
 {%                     % posistion relative to the center of the 'box'
 \begin{tabular}{l}     % content
  \textbf{Отчистка звука}\\
 \end{tabular}
 };

 % {Выделение морфем}
 \node[state,
 below of=soundClear,
 node distance=1.2cm] (morfolize) 
 {%
 \begin{tabular}{l}
  \textbf{Выделение морфем}\\
 \end{tabular}
 };

 % NLP
 \node[state,
 below of=morfolize,
 node distance=1.8cm] (NLP) 
 {%
 \begin{tabular}{l}
  \textbf{NLP}\\
  \parbox{5cm}{Из полученных морфем сделать слова и слова преобразовать в команды}
 \end{tabular}
 };
 
  % DB
 \node[state,
 below of=NLP,
 node distance=1.7cm] (DB) 
 {%
  \begin{tabular}{l}
\textbf{Data Base}\\
 \end{tabular}
 };

 % Play
 \node[state,
 right  of=DB,
 node distance=5cm] (soundPlay) 
 {%
  \begin{tabular}{l}
\textbf{\href{https://www.npmjs.com/package/react-sound}{\textcolor{blue}{React Sound}}}\\
 \end{tabular}
 };


 % draw the paths and and print some Text below/above the graph
 \path (soundGet) edge node[anchor=south,above,text width=4cm]
                   {
                   } (soundClear)
 (soundClear) edge                    (morfolize)
  (morfolize) edge                    (NLP)
(soundGet) edge (recogn)
(recogn) edge (NLP)
(NLP) edge (DB)
(DB) edge (soundPlay)
 ;
\end{tikzpicture}

\end{center}

Рассмотрим каждый из пунктов подробнее:
\begin{enumerate}
\item В первую очередь необходимо с помощью фронтенда научиться получать голосовуху и отправлять ее на сервер. Этим займутся Максим и Матвей и это необходимо научиться делать до хакатона.
\item После разложения сигнала голосовухи в ДФТ получится картинка, подробнее в пункте 3. в секции полезные ссылки, необходимо обучить модель, которая сможет максимально отчистить картринку от шума, довольно подробно о процессе обучения сказано в пункте 2. в секции полезные ссылки. Этим также лучше заняться до хакатона. Этим займется Саша
\item Отчищенный спектр будет подаваться на вход большой и сложной машине, которая выделяет морфемы из спектрограммы, как это делается описано в пункте 2. в секции полезные ссылки. Очень хотелось бы найти готовое решение этой задачи
\item Кажется, что на вход системе будет подаваться ограниченное число команд, выделяется следющие интены:\begin{enumerate}
\item Как мне добраться до название картины / выхода из музея / туалета / раздевалки
\item Расскажи мне про название картины
\item Когда близжайшая экскурсия
\item Какие сейчас работают выставки 
\end{enumerate}

Основная проблема заключается в том, чтобы распознать до какой именно картины хочет добраться пользователь. Ответ на этот вопрос частично дали на \href{https://opendatascience.slack.com/archives/C04N3UMSL/p1568308189002800}{\textcolor{blue}{ODS}}. Этим займутся Булат и Саша, при этом очень прошу \colorbox{red}{Булата} извлечь максимально сути из дискуссии на ODS и переписать эту суть сюда


\end{enumerate}

Важные ссылки:
\begin{enumerate}
\item \href{https://www.youtube.com/watch?v=JpS0LzEWr-4}{\textcolor{blue}{ODS dlcourse.ai}} Не особо подробная лекция, не особо внятный лектор 
\item \href{https://www.youtube.com/watch?v=eke2h9fGtu0}{\textcolor{blue}{Выступление человека из МФТИ}} также в описании к ролику приложена ссылка на \href{https://github.com/nsu-ai-team/voxforge_ru_sphinx_experiments}{\textcolor{blue}{github}}. Необходимо разобраться и потестить как работает их решение
\item \href{https://www.youtube.com/playlist?list=PL0Ks75aof3ThkitsZbUOEQg7Ybl5kB_s3}{\textcolor{blue}{Лекции ФИВТ}} 21, 23, 25. Довольно подробно разбирается теория, также можно посмотреть домашки по курсу доступные в описании под видео

\end{enumerate}

\section{Backend, {Матвей}}

\begin{enumerate}

\item Рассчеты нейронки будут запускаться при вызове метода api, который будет callback ом возвращать рассчеты и результат. Это все сделает Матвей на голом flask. Если кому-то интересно понимать , что будет происходить на бэке \href{https://blog.miguelgrinberg.com/post/the-flask-mega-tutorial-part-i-hello-world}{\textcolor{blue}{Курс грустного мужика}}
\item Ассинхронность 
\item Карта музея и путь от картины А до картины B. Могут возникнуть следующие трудности:
\begin{enumerate}
\item Как передать карту в Front и в каком формате необходимо ее хранить 
\item Как отрисовать путь на карте и отслеживать положение пользователя на этом пути. 
\end{enumerate}
До хакатона нужно найти решение и желательно попробовать его как-то реализовать
\end{enumerate}

\section{Frontend}
\begin{enumerate}
\item Сделать анимацию ожидания при обработке запроса сервером 

\item  Экран формы для прохождения теста за стикеры

\item Всплывающая панелька информации о картинах с возможностью прослушать аудио, полистать фоточки и почитать текст

\item Чатик с ботом, где есть две кнопки:  записать аудио и сфотографировать QR с помощью  \href{https://vk.com/dev/vk_apps_docs}{\textcolor{blue}{VK UI Connect}}
\end{enumerate}
\section{Design + Презентация}
\begin{enumerate}

\item  Подумать над тем как это все будет выглядеть и нарисовать это в какую-нибуть презентацию.


\end{enumerate}
\section{Стикеры}

Если найдутся люди, готовые помочь,  добавляйте их сюда
\begin{enumerate}

\item \href{https://vk.com/sofiamarid_art}{\textcolor{blue}{Софья-Мария}}

\end{enumerate}
\section{Сервер}

\begin{enumerate}

\item Найти сервер минимальной стоимости и запустить на нем что-нибудь простое 

\end{enumerate}

\section*{Успехи}

\begin{enumerate}
\item Получаем QR c фоточки
\item Попробовали React Mic, работает везде, кроме vk mini apps
\end{enumerate}
 
\end{document}